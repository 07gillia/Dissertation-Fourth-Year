\documentclass[12pt,a4paper]{article}
\usepackage{times}
\usepackage{durhampaper}
\usepackage{harvard}
\usepackage{graphicx}
\usepackage{longtable}

\citationmode{abbr}
\bibliographystyle{agsm}

\title{Timeframe Trading Algorithms}
\author{A.L. Gillies}
\student{A.L. Gillies}
\supervisor{M. R. Gadouleau}
\degree{MEng Computer Science}

\date{}
\begin{document}
\maketitle

\section*{Notes for the final paper}

\subsection*{Data}
- taken from a website
- between dates
- in format
- using R and shell scripts managed
- put into a single file
- read in
- initially had an issue with data, not enough
- was rectified and the same steps were followed to put it in the same format
- dumbing of the data was done ? might not mention this but we only take the open price of each minute, thus reducing the dataload, this would not be done in a full scale system as we would have all the metrics and data from the past and all maths that would be used has already been calculated.

\subsection*{Notes}
- ghomas has suggested that we run the code on the GPU as we are running a rolling average
- It could be run on hamilton
- This will take some research but we will continue in the research department first
- Then we will do this parallelism
- then we will implement some of the researched techniques

- the functions we need to implement are:
- when to buy
- when to sell
- how much to buy

- a buy and hold strategy could be implemented first, this would be a buy early in the year and hold until the end, any short term fluctuations would not have any effect but the exact time to buy and sell could be tricky, this in an initial buy at a low price and a late sell at a high price, any fluctuations in the middle are not important only the difference in almost the whole market as this is a conservative approach to trading and will spread the bet.

- genetic trading algorithm, this is more of a technique to make a strategy rather than a strategy itself, the idea is to 'learn' how to trade, using a fitness function and other such techniques on a strategy that is already in place in order to fine tune it. This may be hard to implement but could have some merit.

- there is the possibility of making some machine learning algorithms that will do the same thing, this will have to be researched in R as this is new to me. The techniques are not, but the language in conjunction is.

- some hands on trading strategies could be considered, just with the algorithm being the active party not the person. These include position trading, swing trading, scalping, and day trading. These can be considered.

- alpha is something that may be useful, it is a metric used to measure risk. Alpha is one of five technical risk ratios; the others are beta, standard deviation, R-squared, and the Sharpe ratio. All these will be looked into, to see if any benefit can be drawn from them. They may be useful in later calculations.

\end{document}