\documentclass[12pt,a4paper]{article}
\usepackage{times}
\usepackage{durhampaper}
\usepackage{harvard}
\usepackage{graphicx}
\usepackage{longtable}

\citationmode{abbr}
\bibliographystyle{agsm}

\title{Timeframe Trading Algorithms}
\author{A.L. Gillies}
\student{A.L. Gillies}
\supervisor{M. R. Gadouleau}
\degree{MEng Computer Science}

\date{}
\begin{document}
\maketitle

\section*{Literary Review}

\subsection*{Books}

\subsubsection*{Electronic and Algorithmic Trading Technology - The Complete Guide}
By Kendall Kim \newline
Started Reading - 30/09/2017 \newline
Finished Reading -  \newline
\newline
- This book was found in Durham library using the search term "algorithmic trading". \newline
- The first chapter is a fairly outdated history of algorithmic trading and goes through the terms that will be used in the book. It also goes through how algorithmic trading was used in the late 2000s, namely to break up large buy or sell orders so as to reduce the effect that this has on the valuation of the stock. \newline
- The second chapter comes from the point of view of a manager that covers all aspects of a trade. Pre-trade, the trade itself, and post trade. The signing of contracts and other such requirements. There is also discussion of data speed. This chapter is more around the theory and execution of the environment around the algorithms, it also touches on the shift of usage of brokers from over the phone to more through online. \newline
- Chapter three covers the adoption and growth of algorithmic trading. Hedge funds especially have pushed this forwards. \newline
- Chapter four, the repeal of Rule 390 and the consolidation of stock exchanges is discussed. The consolidation has a basis in speeding up technology and business development. With the repeal of rule 390 being a key factor in the acceleration of this. \newline
- Chapter five, very useful in real life, we only implemented paper portfolio logic. TWAP and AWAP are considered. \newline
-

\subsection*{Papers}



\end{document}